\pagebreak

\section{Fazit}
Dadurch dass es diverse Angriffsmöglichkeiten und Gegenmaßnahmen gibt,
ist es sowohl für Angreifer als auch Angriffsziel nicht einfach,
eine Übersicht über die offenen Schwachstellen eines Prozesses zu halten.
Wie bereits in \autoref{sec:wellknown} behandelt, gibt es sehr viele Buffer Overflow
angriffe in der welche inzwischen Bekannt sind. Dadurch das ein Angreifer jedoch
wie in \autoref{sec:bounties} erwähnt root-zugriff erlangen kann, ist es für einen
versierten Angreifer möglich, die Spuren des Angriffs auf dem Zielsystem zu
entfernen. Natürlich können nicht ohne weiteres Logs von einem Zwischenserver
oder dem Internetanbieter gelöscht werden - trotzdem werden viele der Angriffe
meist nicht - oder nicht direkt - als solche erkannt und stellen damit eine der
grössten Bedrohungen im Bereich der Cybersecurity dar.

%☑️Themen: Kap 2 Bekannte -> Dadurch relevant   
%Themen: Kap 3.3 3.4 -> Referenz(Es gibt mehrere Arten)
%☑️Themen: Kap 4.2 Erlangen von Zugriff
%Themen: Kap 5.6 Exploit
%Themen: Kap 6.1 Übersicht
%☑️Themen: Kap 6.8.1 Relevanter und lukrativer Bereich



















