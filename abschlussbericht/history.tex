\section{Geschichte}
\subsection{Bekannte Buffer Overflows} \label{sec:wellknown}
Es folgen Beispiele für historische sowie aktuelle Angriffe auf der Grundlage von Buffer Overflows:

„The Morris Worm“ der am 2. November 1988 ins damals noch junge Internet freigelassen wurde und sich rasant verbreitete,
verursachte großen Schaden in Form von überlasteten Systemen und Totalausfällen. Der Wurm wurde vom amerikanischer Robert T. Morris in C geschrieben
und umfasste ca. 3200 Programmierzeilen. Morris war Student an der Cornell University und wollte mit seinem Programm die angeschlossenen Rechner an einem Netz zählen.
Stattdessen legte er nach nur 15 Stunden 10\% des damaligen Internets lahm. Morris wurde zu drei Jahren Haft und einer Geldstrafe von 10.000 US-Dollar verurteilt. \cite{wiki1}

Ein weiteres historisches Beispiel ist der „SQL-Slammer“. Dieser wurde am 25. Januar 2003 entfesselt und hatte schon innerhalb von 30 Minuten über 75.000 Opfer.
Der Computerwurm infizierte ungepatchte Microsoft SQL Server 2000 und nutze dabei zwei Buffer Overflow Schwachstellen.
Das Besondere an diesem Wurm war seine kompakte Größe: Er bestand aus einem UDP-Paket von lediglich 376 Bytes und bewegte sich ausschließlich im Arbeitsspeicher des befallenen
Computers, nicht jedoch auf der Festplatte. Der Wurm lieferte dabei keinerlei Payload, sondern versuchte lediglich sich selbst zu kopieren und
so viele Computer wie möglich zu infizieren. Bei den Entwicklern handelte es ich um zwei Mitglieder der Gruppe 29A, die im Jahr 2004 gefasst wurden. \cite{wiki2}

Eine aktuellere Schwachstelle fand sich 2019 im Messenger Dienst WhatsApp. Diese ermöglichte es Angreifern,
mit der Hilfe von manipulierten Videodateien Malware einzuschleusen und sich so Zugriff auf Smartphones zu verschaffen.
Die Sicherheitsabteilung von Facebook sprach von einem „Stack-based Buffer Overflow“ der über korrupte MP4-Dateien ausgenutzt wurde.
Der Fehler wurde durch einen Patch behoben. \cite{whatsapp1}

Beim letzten Beispiel handelt es sich um eine Sicherheitslücke in HP-Druckern.
Dabei dreht es sich konkret um zwei Sicherheitsprobleme in der Firmware von HP, von der viele Druckermodelle betroffen sind.
Eine Liste dieser wurde bereits veröffentlicht. Die Sicherheitslücken erlauben dem Angreifer durch veränderte Anfragen an den Drucker einen Buffer Overflow auszulösen
und Schadcode zu injizieren. Die Lücken wurden mittlerweile durch ein Update geschlossen, es existieren jedoch immer noch viele anfällige Systeme. \cite{hpvuln}
