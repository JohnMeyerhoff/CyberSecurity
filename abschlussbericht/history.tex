\section{Geschichte}
Nachdem der Begriff Buffer Overflow und die verschiedenen Typen erklärt worden sind,
werden jetzt zwei ältere, aber sehr bekannte Buffer-Overflow-Attacken und zwei aktuell relevante Beispiele gezeigt,
die dieselbe Schwachstelle ausnutzen.

\subsection{Bekannte Buffer Overflows}

Beim ersten handelt es sich um „The Morris Worm“ der am 2. November 1988 ins damals noch junge Internet freigelassen
worden ist und sich rasant verbreitete. Er verursachte hohen Schaden in Form von überlasteten Systemen bzw. Ausfällen
von Systemen. Das Programm wurde vom amerikanischen Robert T. Morris in der Sprache C geschrieben, dessen Dateigröße
ca. 3200 Programmierzeilen umfasst. Robert T. Morris war Student an der Cornell University und die eigentliche
Idee hinter seinem Experiment war es, die angeschlossenen Rechner an einem Netz zu zählen. Stattdessen wurden alle
15 stunden 2000 Unix Systeme infiziert. Im Endeffekt war damals 10\% des gesamten Internets betroffen und um den
Schaden des Virus einzudämmen, wurden regionale Netzwerke vom Internet getrennt.

Der nächste von den älteren ist der „SQL-Slammer“. Dieser begann am 25. Januar 2003 und hatte schon innerhalb
von 30 Minuten 75.000 Opfer. Der Computerwurm infizierte ungepatchte Microsoft SQL Server 2000 und nutze zwei
Buffer Overflows. Das Besondere an diesem Wurm war seine Größe: Er bestand nur aus einem UDP-Paket mit lediglich
376 Bytes und befand sich nur auf dem Arbeitsspeicher eines Computers und nicht jedoch auf der Festplatte.
Zusätzlich enthielt er keine Payload und seine einzige Aufgabe war es, sich selbst zu kopieren und auf so
vielen zufälligen Computern wie möglich zu verbreiten. Bei den Angreifern handelt es ich um zwei Mitglieder
der Virenschreibergruppe 29A, die im Jahr 2004 gefasst worden sind.

Der Erste der neuen Attacken geschah bei einem der größten Messenger, und zwar WhatsApp. im Jahr
2019 wurde eine Sicherheitslücke ausgenutzt, die in DoS-Attacken über manipulierten MP4-Dateien endete.
Mit der Hilfe der Videos konnte man sich Zugriff auf Smartphones verschaffen und Malware einschleusen.
Die Security Abteilung sprach von einem „Stack-based buffer Overflow“ der über korrupte MP4-Dateien ausgenutzt wird.
Der Fehler wurde durch einen neuen Patch behoben und es wird empfohlen, immer die neueste Version zu haben.

Beim letzten Angriff handelt es sich um eine Sicherheitslücke bei HP-Druckern. Dabei handelt es sich konkret um zwei
Sicherheitsprobleme in der Firmware von HP und viele Druckermodelle sind davon betroffen. Eine Liste dieser wurde
bereits veröffentlicht. Die Sicherheitslücke hat zur Folge, dass der Angreifer eine korrupte Datei an den Drucker
sendet und diese zu einem Buffer Overflow führt, dadurch haben die Angreifer dann Fernzugriff auf das Gerät und können
infizierten Code ausführen. Auch hier wurde die Lücke durch ein Update geschlossen.



