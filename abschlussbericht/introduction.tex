\section{Abstract}
Buffer Overflows gehören trotz ihres hohen Alters noch immer zu den relevantesten Schwachstellen in Computerprogrammen.
Aus diesem Grund ist es unabdinglich für jeden, der sich im Feld der Software-Entwicklung oder IT-Sicherheit bewegt, 
ein grundlegendes Verständnis für Buffer Overflows aufzubauen. Ein Überblick über große und aktuelle Angriffe zeigt, 
wie verheerend die Auswirkungen einer solchen Schwachstelle sein können. Ein Buffer Overflow beschreibt im weitesten 
Sinne das “Überlaufen” eines Speicherbereiches durch unvorhergesehene Eingaben, wodurch Schadcode in einen laufenden 
Prozess injiziert und ausgeführt werden kann.

Der vorliegende Projektbericht beschäftigt sich deshalb mit einer theoretisch-technischen Einführung sowie der praktischen 
Analyse von Buffer-Overflow-Schwachstellen. Um eine Grundlage für praktische Tests zu schaffen, wurden zunächst  
die Struktur und der Ablauf eines Programms im Speicher analysiert und theoretische Angriffsmöglichkeiten konstruiert.
Anschließend wurde ein fiktives Angriffsszenario und ein verwundbares C-Programm entwickelt. Hierbei zeigte sich schnell,
dass bereits die Nutzung von vermeintlich harmlosen Funktionen, wie \codeline{fprint()} oder \codeline{gets()}, schwerwiegende Auswirkungen
auf die Sicherheit einer Applikation haben kann. Um die zuvor konstruierten Angriffstechniken real zu erproben und die
Sicht eines Angreifers möglichst realistisch zu analysieren, wurde das verwundbare Programm anschließend im GNU-Debugger
durchleuchtet. Dabei ließ sich klar erkennen, dass der Angreifer von einer Kopie des anzugreifenden Programms, oder sogar
des Source Codes, profitiert. Mit dem aus dem Debugger erlangten Wissen wurde nun ein Exploit gebaut und in einer
simulierten Server-Umgebung ausgeführt. Unter Zuhilfenahme eines injizierten Assembler Programms, konnte auf dem
Zielsystem erfolgreich eine privilegierte Shell geöffnet werden.
Abschließend wurde sich mit den wichtigsten
Abwehrmechanismen auseinandergesetzt und es wurden “cutting-edge” Präventions-Mechanismen,
wie statische Codeanalyse oder Canaries, untersucht. Hier zeigte sich klar,
dass einem Angreifer die Arbeit zwar erschwert werden kann,
Buffer Overflows jedoch nie vollständig verhindert werden können.

Durch diese auf praktische Beispiele fokussierte Herangehensweise soll der Leser dieses Projektberichts die Thematik
der Buffer Overflows besser verstehen und einen direkten Nutzen für seine Arbeit ziehen können.
\pagebreak

\section{Grundaufbau}
    \begin{itemize}
        \item Eingabemöglichkeit
        \item Speichern der Eingabe
        \item Ablegen von Anweisungen durch \underline{übergroße Eingabe}  
        \item Ausführen der Anweisungen $\rightarrow$ Remote Code Execution
    \end{itemize}
