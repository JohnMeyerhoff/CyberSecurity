\section{Abstract}
Buffer Overflows sind trotz ihres hohen Alters noch immer eine der, 
wenn nicht sogar die relevanteste, Schwachstelle in Computerprogrammen. 
Aus diesem Grund ist es unabdinglich für jeden der sich im Feld der Software-Entwicklung 
oder IT-Sicherheit bewegt, ein Grundlegendes Verständnis für Buffer Overflows aufzubauen. 
Ein Überblick über große und aktuelle Angriffe zeigt, wie verheerend die Auswirkungen einer 
solchen Schwachstelle, in den falschen Händen, sein kann. 

Der vorliegende Projektbericht beschäftigt sich deshalb, mit einer theoretisch, 
technischen Einführung, sowie praktischen Analyse von Buffer Overflow Schwachstellen. 
Ein Buffer Overflow beschreibt hierbei im Kern, das “Überlaufen” eines Speicherbereiches 
durch unvorhergesehene Eingaben, wodurch Schadcode in einen laufenden Prozess injiziert 
und ausgeführt werden kann.
Um eine Grundlage für praktische Tests zu schaffen, wurde zunächst  die Struktur und der 
Ablauf eines Programms im Speicher analysiert und theoretische Angriffsmöglichkeiten 
konstruiert. Anschließend wurde ein fiktives Angriffsszenario und ein verwundbares 
C-Programm entwickelt. Hierbei zeigte sich schnell, das bereits die Nutzung von vermeintlich 
harmlosen Funktionen, wie fprint() oder gets(), schwerwiegende Auswirkungen auf die 
Sicherheit einer Applikation haben kann. Um die zuvor konstruierten Angriffstechniken, 
real zu erproben und die Sicht eines Angreifers möglichst realistisch zu analysieren, 
wurde das verwundbare Programm anschließend im GNU-Debugger durchleuchtet. Dabei ließ sich 
klar erkennen, dass der Angreifer, von einer tatsächlichen Kopie des anzugreifenden 
Programms, wenn nicht sogar des Source-Codes, profitiert. Mit dem aus dem Debugger 
erlangten Wissen, wurde nun ein Exploit gebaut und in einer simulierten Server-Umgebung 
ausgeführt. Mit der Zuhilfenahme eines injizierten Assembler Programms, konnte auf dem 
Zielsystem erfolgreich eine privilegierte Shell geöffnet werden.
Abschließend wurde sich mit den wichtigsten Abwehrmechanismen von Compiler und 
Betriebssystem auseinandergesetzt, sowie “cutting-edge” Präventions-Mechanismen, 
wie KI gestützte Codeanalyse, untersucht. Hier zeigte sich klar, dass einem Angreifer 
die Arbeit zwar erschwert werden kann, Buffer Overflows jedoch nie vollständig verhindert 
werden können.

Durch diese auf praktische Beispiele fokussierte Herangehensweise, sollte der Leser 
dieses Projektberichts die Thematik der Buffer Overflows besser verstehen und einen 
direkten Nutzen für seine Arbeit ziehen können.
\pagebreak

\section{Grundaufbau}
    \begin{itemize}
        \item Eingabemöglichkeit
        \item Speichern der Eingabe
        \item Ablegen von Anweisungen durch \underline{übergroße Eingabe}  
        \item Ausführen der Anweisungen $\rightarrow$ Remote Code Execution
    \end{itemize}
