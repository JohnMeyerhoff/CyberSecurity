\section{Einleitung}
Aktuelle Statistiken zeigen, dass Buffer Overflows noch immer zu den relevantesten Schwachstellen in Computerprogrammen gehören.
Jeder, der sich im Bereich der Informationstechnik und im Besonderen in der Anwendungsentwicklung oder IT-Sicherheit bewegt,
sollte ein grundlegendes Verständnis für diese Schwachstellen aufbauen.

Der folgende Projektbericht beschäftigt sich daher mit der Theorie und Anwendungspraxis hinter Angriffen auf der Basis von Buffer Overflows.
Der Leser soll verstehen, warum die Gefahr durch Buffer Overflows noch immer hoch ist und wie er sich möglichst effizient schützen kann.
Hierfür werden zunächst einige historische sowie aktuelle Beispiele für Angriffe mit Buffer Overflows betrachtet und ihre Auswirkungen dargelegt.
Anschließend werden die theoretisch technischen Grundlagen für eine tiefere praktische Analyse gelegt. 
An konkreten Beispielen werden Shellcode und verschiedene Angriffstechniken erläutert. 
Diese werden darauf in einer simulierten Serverumgebung ausgeführt und ein reales System kompromittiert. 
Abschließend werden unterschiedliche Abwehrmechanismen analysiert und erklärt.

Durch eine umfangreiche Betrachtung von Buffer Overflows aus der Sicht eines Angreifers und die Erläuterung verschiedener Verteidigungsmaßnahmen, 
sollte der Leser ein besseres Verständnis für die Thematik bekommen und für Angriffe dieser Art sensibilisiert werden. \cite{owasptop}
