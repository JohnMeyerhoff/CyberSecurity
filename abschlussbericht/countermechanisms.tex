\pagebreak
\section{Gegenmaßnahmen}
Da sich einem Angriff durch Buffer Overflow mit der passenden Payload
ein Shellcode ausführen lässt, welcher dann eine Root Shell
öffnen kann, gehören
Buffer-Overflow-Angriffe zu den gefährlichsten.

Wie in \autoref{sec:explainersub} bereits aufgeführt,
gehören zu einem Buffer-Overflow-Angriff mehrere kombinierte Teile. Wenn
man nun verhindern möchte, dass ein Programm über diesen Angriff kompromittiert wird, 
so hat man mehrere Möglichkeiten, diese Teile oder ihre Kombination aufzuhalten. \cite{Werthmann2006SurveyOB}\\

Es folgen mehrere Optionen, grob nach Aufwand (für den Entwickler) sortiert.
\subsection{Übersicht der Maßnahmen}
Low-Level:
\begin{itemize}
    \item Hardware-basierte Lösungen
    \item Betriebssystembasierte Ansätze
\end{itemize}
Passive Härtung der Programme:
\begin{itemize}
    \item C Range Error Detector und Out Of Bounds Object
    \item Address Space Layout Randomization
    %\item Safe Pointer Instrumentalisierung wäre zu viel https://download.vusec.net/papers/delta-pointers_eurosys18.pdf
    \item Manuelles Buffer-Overflow-Blocken (Input-Bereinigung)
\end{itemize}
Aktive (analysierende) Lösungen:
\begin{itemize}
    \item Statische Analyse
    \item Stack-Schutz mit ''Canary'' (Zufallszahl)
\end{itemize}

\subsection{Low-Level-Probleme}
Sowohl Hardwarelösungen als auch betriebssystembasierte Lösungen haben
das grundlegende Problem, dass die Verhinderung von Buffer Overflows zu
ungewünschten Nebeneffekten führen kann. Es ist auf jeden Fall möglich,
jegliche Overflows zu verhindern - dabei können jedoch auch vom
Entwickler gewünschte Overflows blockiert werden, sodass Programme nicht mehr
ordnungsgemäß funktionieren. Manchmal wird aus Gründen der Effizienz ein
Buffer zum Überlaufen gebracht, ohne dass dieser Überlauf unkontrolliert
geschieht. Leider ist es praktisch nicht umsetzbar, in einer Hardwarelösung
zu unterscheiden, welcher Buffer Overflow böswillig und welcher gewollt ist. 
Damit ist dieser Ansatz nicht praktikabel.
\pagebreak

\subsection{C Range Error Detector und Out Of Bounds Object}
Ein Out Of Bounds Object ist eine vereinfachte Lösung, um Referenzen ungefährlich zu machen.
Es wird verhindert, dass auf Speicher außerhalb des Programms zugegriffen wird, indem jede
Adresse, welche nicht im spezifizierten Bereich liegt, auf ein bestimmtes Objekt, das sogenannte
''Out Of Bounds Object'' verweist. Diese Methode ist nicht gängig, da sie leicht
umgangen werden kann, sofern der Angreifer weiß, welcher Speicherbereich für das Programm
vorgesehen ist.



\subsection{Testen}
Es gibt mehrere Möglichkeiten, um kompilierte Programme auf ihre Sicherheit
und Robustheit zu testen. Wartbarkeit und Erweiterbarkeit
sind langfristig ebenfalls zu beachten, da es bei Änderungen am Quellcode
zu Fehlern kommen kann, welche Schwachstellen mit sich bringen.
Mit Werkzeugen wie SonarLint können vor allem häufig auftretende Fehler entdeckt
werden.
Im Bereich der Overflow Payloads gibt es mehrere Werkzeuge, um Fuzzy Testing
zu betreiben. Bei Fuzzy Tests wird gezielt-zufällig auf eine
potentielle Schwachstelle getestet, wobei die tatsächlichen Aufrufe
von einem sogenannten Fuzzer erstellt werden.
Als Alternative zum Fuzzing gibt es spezifische Payloads und
Escape-Sequenzen mit denen - auch automatisiert - getestet werden kann.

\subsubsection{Statische Analyse}
Wenn das Programm bereits vor der Ausführung analysiert wird, kann ein
Tool bestimmen, an welchen Stellen Schwachstellen vorhanden sind, und
ggf. vorschlagen, wie diese behoben werden können. Leider ist bei größeren
Projekten die Abwesenheit von Schwachstellen unmöglich zu bestimmen. 
Daher ist dieser Ansatz zeitsparend, aber durch die fehlende finale Gewissheit alleine nicht ausreichend.


\subsubsection{Bug-Bounty-Programme} \label{sec:bounties}
Viele Unternehmen externalisieren zusätzliche Tests durch Prämien für gefundene Sicherheitslücken.
Die Prämien sind meist davon abhängig, in welcher Version der Software der Angriff 
wirksam ist und welche Voraussetzungen erfüllt sein müssen (physikalischer Zugriff / Netzwerkverbindung),
um den Angriff auszuführen. Auch wichtig für die Höhe der Prämie ist, welche Art des Zugriffes
der Angriff ermöglicht.

Wurde ein Buffer Overflow entdeckt, so wird meist die höchste Prämienstufe ausgezahlt, da bei einem
Buffer Overflow mit der passenden Payload ein Shellcode ausgeführt werden kann, welcher dann das
höchste Ziel bei einem Angriff - eine Root Shell - erreicht.
Diese hohe Zugriffsstufe macht einen Buffer-Overflow-Angriff immer zu einem der attraktivsten Vektoren. 
Sind mehrere Schwachstellen bekannt, ist diese Version meist veraltet. 


\subsection{Stack-Schutz mit Canaries} \label{sec:canaries}
Die Bezeichnung Canary (engl. Kanarienvogel) leitet sich ab aus der Verwendung von Kanarienvögel als
Indikator für Gas in Minen. Die Canaries im Code werden als Stack-Schutz verwendet. Das bedeutet,
dass beispielsweise Zufallszahlen im Programm auf dem Stack sind und bei einem Buffer-Overflow-Angriff
überschrieben werden. Ein Tool wie StackGuard kann in diesem Fall anhand der Änderung einen Fehler feststellen und
die Ausführung des Programms abbrechen. Dies stellt offensichtlich eine effektive Möglichkeit dar,
Buffer Overflows zu erkennen.

%TODO: ASLR Beschreiben
%ASLR wird ausgehebelt über Adress-Wissen
%ref auf debugging
\subsection{Address Space Layout Randomization} \label{sec:aslr}
Bei \acrlong*{aslr} (\acrshort*{aslr}) geht es vor allem darum, einem potentiellen Angreifer zu erschweren,
auf Adressen im Stack zuzugreifen, bzw. zu wissen, was sich dort befindet.
\gls*{aslr} ist leider, wie in \autoref{sec:explainersub} dargestellt, nicht effektiv, wenn der Angreifer
bereits eine Adresse im anzugreifenden Prozess kennt.
ASLR ist bei Linux eine im Kernel bereitgestellte Funktionalität.
Wenn in \codeline{sysctl} die Option \codeline{{kernel.randomize\_va\_space=0}} gesetzt ist,
wird \gls*{aslr}  nicht verwendet. \cite{aslrandrew} \\
In neueren Distributionen ist \gls*{aslr} standardmäßig eingeschaltet.
Die Speicheradressen von eingebundenen Bibliotheken und Methoden werden
zufällig gewählt, sodass wichtige Komponenten, wie beispielsweise die
Prüfung einer Lizenz, nicht immer am gleichen Offset liegen.

Compiler haben ein Pendant zu \gls*{aslr}, welches \gls*{pie}
genannt wird. So müssen also beim GCC zum Kompilieren die Flags \codeline{gcc -fPIE -pie}
gesetzt werden. Dies führt dann dazu, dass das kompilierte Programm auch tatsächlich an
unabhängigen und von Ausführung zu Ausführung unterschiedlichen Speicheradressen lauffähig
ist.
Nach dem gleichen Konzept ist es dann einem Angreifer nicht ohne weiteres möglich, die
Speicheradressen der Komponenten eines Programms zu kennen, da diese abhängig
von der (nun zufälligen) Speicheradresse des Programms sind. Kombiniert man nun beide Lösungen,
so sind nun alle Adressen zufällig und nicht voneinander abhängig.

\subsection{Manuelles Buffer-Overflow-Blocken}
Ein Programmierer kann Buffer Overflows verhindern, indem er die Eingaben,
welche er entgegennimmt, zuerst filtert und dann validiert. Leider funktioniert
dies nicht immer zuverlässig und ist für einige Angriffsvektoren schlichtweg nicht
möglich, da oft nicht zwischen normaler und böswilliger Anfrage unterschieden werden kann. 
Es ist aber dennoch hilfreich, die sogenannte Input Sanitization einzubauen.
Input Sanitization befasst sich grundlegend damit, infizierte Eingaben zu bereinigt, 
sodass im schlimmsten Fall eine semantisch inkorrekte Eingabe weiterverarbeitet
wird, nicht aber Datenlecks oder ungewollte Aufrufe entstehen. \cite{sanitize} \\
Dadurch ist dieser Ansatz sehr effektiv, aber dafür auch sehr aufwändig.




