\pagebreak
\section{Gegenmaßnahmen}
Wie in \autoref{sec:explainersub} bereits aufgeführt,
gehören zu einem Buffer-Overflow-Angriff mehrere kombinierte Teile. Wenn
man nun verhindern möchte, dass ein Programm über diesen Angriff gestürzt werden
kann, so hat man mehrere Möglichkeiten diese Teile aufzuhalten oder die Kombination
zu blockieren. \cite{Werthmann2006SurveyOB} Es folgen mehrere Möglichkeiten, grob nach Aufwand (für den Entwickler) sortiert.
\subsection{Übersicht der Maßnahmen}
Low-Level:
\begin{itemize}
    \item Hardware-basierte Lösungen
    \item Betriebssystembasierte Ansätze
\end{itemize}
Passive Härtung der Programme:
\begin{itemize}
    \item C Range Error Detector und Out Of Bounds Object
    \item Address space layout randomization
    %\item Safe Pointer Instrumentalisierung wäre zu viel https://download.vusec.net/papers/delta-pointers_eurosys18.pdf
    \item Manuelles Buffer Overflow Blocken (Input-Bereinigung)
\end{itemize}
Aktive (Analysierende) Lösungen:
\begin{itemize}
    \item Statische Analyse
    \item Stack-Schutz mit ``Canary'' (Zufallszahl)
\end{itemize}

\subsection{Low-Level Probleme}
Sowohl Hardwarelösungen als auch Betriebssystembasierte Lösungen haben
das grundlegende Problem, dass die Verhinderung von Buffer-Overflows zu
ungewünschten Nebeneffekten führen kann. Es ist auf jeden Fall möglich,
jegliche Overflows zu verhindern - dabei werden jedoch auch vom
Entwickler gewünschte Overflows verhindert, sodass Programme nicht mehr
ordentlich funktionieren. Manchmal wird aus gründen der Effizienz ein
Buffer zum Überlaufen gebracht, ohne dass dieser Überlauf unkontrolliert
ist. Leider ist es Praktisch nicht umsetzbar, in einer Hardwarelösung
zu entscheiden, welcher Buffer-Overflow böswillig ist.

\subsection{C Range Error Detector und Out Of Bounds Object}
Ein Out Of Bounds Object ist eine Vereinfachte Lösung um Referenzen ungefährlich zu machen.
Es wird verhindert, dass auf Speicher außerhalb des Programms zugegriffen wird, indem Jede
Adresse welche nicht im spezifizierten Bereich liegt auf ein bestimmtes Objekt, das sogenannte
"Out Of Bounds Object" verweist. Dadurch kann innerhalb des laufenden Programms erkannt werden, das
etwas falsch gelaufen ist. Diese Methode ist nicht gängig, da sie technisch gesehen
umgangen werden kann, solange der Angreifer weiß, welcher Speicherbereich für das Programm
vorgesehen ist.

\subsection{Statische Analyse}
Wenn das Programm bereits vor der Ausführung analysiert wird, kann ein
Tool bestimmen, an welchen stellen Schwachstellen vorhanden sind und
ggf. vorschlagen, wie diese behoben werden können. Leider ist bei größeren
Projekten die Aussage, dass keine Schwachstellen mehr vorhanden seien
unmöglich zu treffen.

\subsection{Stack-Schutz mit Canaries} \label{sec:canaries}
Die Bezeichnung Canary (Engl. Kanarienvogel) stammt aus dem Gebrauch der Kanarienvögel als
Indikator für Gas in Mienen. Die Canaries im Code werden als Stack-Schutz verwendet. Das bedeutet,
dass beispielsweise Zufallszahlen im Programm auf dem Stack sind und bei einem Buffer-Overflow-Angriff
überschrieben werden. Ein Tool wie StackGuard kann in diesem Fall anhand der Änderung einen Fehler feststellen und
die Ausführung des Programms abbrechen.

%TODO: ASLR Beschreiben
%ASLR wird ausgehebelt über Adress-Wissen
%ref auf debugging
\subsection{Address space layout randomization} \label{sec:aslr}
Bei \gls{aslr} geht es vor Allem darum, einem potentiellen Angreifer zu erschweren
auf Adressen im Stack zuzugreifen bzw. zu wissen was sich dort befinden müsste.
\gls{aslr} ist leider wie in \autoref{sec:explainersub} dargestellt nicht effektiv, wenn der Angreifer
bereits eine Adresse kennt.
ASLR ist bei Linux eine im Kernel bereitgestellte Funktionalität.
Wenn in \codeline{sysctl} die option \codeline{{kernel.randomize\_va\_space=0}} gesetzt ist,
wird \gls{aslr}  nicht verwendet. \cite{aslrandrew} In neueren Distributionen ist \gls{aslr} 
standardmäßig eingeschaltet.
Die Speicheradressen von eingebundenen Bibliotheken und Methoden werden
zufällig gewählt, sodass wichtige Komponenten wie beispielsweise die
Prüfung einer Lizenz nicht immer am gleichen Offset liegen. Dann könnte ein Angreifer
mit dem Wissen über die Speicheradresse des Programms die Speicheradresse beliebiger
Komponenten zuverlässig errechnen und erreichen.

Compiler haben ein Pendant zu \gls{aslr} welches \gls{pie}
genannt wird. So müssen also bei GCC zum Kompilieren die Flags \codeline{gcc -fPIE -pie}
gesetzt werden. Dies führt dann dazu, dass das Kompilierte Programm auch tatsächlich an
unabhängigen und von Ausführung zu Ausführung unterschiedlichen Speicheradressen lauffähig
ist.
Nach dem gleichen Konzept ist es dann einem Angreifer nicht ohne weiteres möglich, die
Speicheradressen der Komponenten eines Programms zu wissen da diese Abhängig
von der (nun zufälligen) Speicheradresse des Programms sind. Kombiniert man nun beide Lösungen,
so sind nun Alle Adressen zufällig und nicht voneinander abhängig.

\subsection{Manuelles Buffer Overflow Blocken}
Ein Programmierer kann Buffer Overflows verhindern, indem er die Eingaben
welche er entgegennimmt zuerst filtert und dann validiert. Leider funktioniert
dies nicht immer zuverlässig und ist für manche Angriffsvektoren schlichtweg nicht
möglich, da es teilweise nichts gibt um zwischen normaler und böswilliger Anfrage zu
unterscheiden. Es ist aber dennoch hilfreich die sogenannte Input Sanitization einzubauen.
Input Sanitization befasst sich grundlegend damit, dass infizierte Eingaben desinfiziert
werden, sodass im schlimmsten fall eine semantisch inkorrekte Eingaben weiterverarbeitet
werden, nicht aber Datenlecks oder ungewollte Aufrufe entstehen. \cite{sanitize} 

\subsection{Testen}
Es gibt mehrere Möglichkeiten um kompilierte Programme auf ihre Sicherheit
und Robustheit zu testen. Diese beiden Qualitätsmerkmale sind vor Allem
im Bezug zu Buffer-Overflows am wichtigsten. Wartbarkeit und Erweiterbarkeit
sind langfristig auch zu beachten, da es bei Änderungen am Quellcode
zu Fehlern kommen kann, welche Schwachstellen herbeibringen.
Mit Werkzeugen wie Sonar lint können vor allem häufig auftretende Fehler entdeckt
werden.
Im Bereich der Overflow Payloads gibt es mehrere Werkzeuge um Fuzzy Testing
zu betreiben. Beim Fuzzy Test wird strukturiert zufällig auf eine
potentielle Schwachstelle getestet, wobei die tatsächlichen aufrufe
von einem sogenannten Fuzzer erstellt werden.
Als Alternative zum Fuzzing gibt es Spezifische Payloads und
Escape-Sequenzen welche - auch automatisiert - getestet werden können.

\subsubsection{Bug Bounty Programme} \label{sec:bounties}
Viele Unternehmen externalisieren zusätzliche Tests durch Prämien für gefundene Sicherheitslücken.
Die Prämien sind meist davon abhängig, in welcher Version der Software der Angriff 
wirksam ist und welche Voraussetzungen erfüllt sein müssen (physikalischer Zugriff / Netzwerkverbindung)
um den Angriff auszuführen. Auch wichtig für die Höhe der Prämie ist, welche art des Zugriffes
der Angriff ermöglicht.

Wurde ein Buffer Overflow entdeckt, so wird meist die höchste Prämienstufe ausgezahlt, da bei einem
Buffer Overflow mit der richtigen Payload ein Shellcode ausgeführt werden kann welcher dann das
höchste Ziel bei einem Angriff - eine Root Shell - erzielt.
Diese hohe Zugriffsstufe macht einen Buffer Overflow angriff immer zum Attraktivsten Angriffsvektor,
falls mehrere Schwachstellen einer Software bekannt sind. Sind mehrere Schwachstellen bekannt, ist die
Version aber meist veraltet. 




