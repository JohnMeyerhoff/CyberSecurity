\begin{frame}{Praktische Analyse}{Programmierfehler}
    Overflow-anfällig sind Sprachen, welche direkte Zugriffe auf die Speicherstrukturen\\ des Systems ermöglichen:
    \begin{itemize}
        \vspace{1em}
        \item Assembly
        \item C/C++
        \item Fortran
    \end{itemize}
\end{frame}

\begin{frame}{Praktische Analyse}{Programmierfehler}
    Problematisch sind Funktionen, welche
    keine Kontrolle auf die Länge des Inputs implementieren: %ausüben / durchführen
    \begin{itemize}
        \vspace{1em}
        \item \codeline{gets(buffer)}\\ Erwartet Input und kopiert diesen in den angegebenen Speicher
        \vspace{1em}
        \item \codeline{strcopy(buffer, input)}\\ Kopiert einen Input
        in den angegebenen Speicher
    \end{itemize}
\end{frame}


\begin{frame}{Praktische Analyse}{Format-String-Schwachstelle}
    Unvorsichtige Verwendung von Formatierungsfunktionen:\\ %Leichtsinnige
    \codeline{printf(``\%s'', chars)}\\ Ist eine gute Umsetzung,\\
    \codeline{printf(``\%s'', chars)}\\ hingegen nicht.
%BIG CODEIMAGE
\end{frame}


\begin{frame}{Praktische Analyse}{Demonstration}
    [Screenshare]
\end{frame}

\begin{frame}{Praktische Analyse}{Demonstration}
    [Bild1]
\end{frame}

\begin{frame}{Praktische Analyse}{Demonstration}
    [Bild2]
\end{frame}

