\pagebreak
\begin{frame}{Gegenmaßnahmen}{Übersicht}
    Low-Level:
\begin{itemize}
    \item Hardware-basierte Lösungen
    \item Betriebssystembasierte Ansätze
\end{itemize}
Passive Härtung der Programme:
\begin{itemize}
    \item C Range Error Detector und Out Of Bounds Object
    \item Address Space Layout Randomization
    %\item Safe Pointer Instrumentalisierung wäre zu viel https://download.vusec.net/papers/delta-pointers_eurosys18.pdf
    \item Manuelles Buffer-Overflow-Blocken (Input-Bereinigung)
\end{itemize}
Aktive (analysierende) Lösungen:
\begin{itemize}
    \item Statische Analyse
    \item Stack-Schutz mit ``Canary'' (Zufallszahl)
\end{itemize}
\end{frame}



\begin{frame}{Gegenmaßnahmen}{Low-Level-Probleme}
    Hardware- und betriebssystembasierte Lösungen
   \begin{itemize}
       \item können alle Overflows verhindern
       \item können nicht unterscheiden ob böswillig oder geplant
   \end{itemize}
    
    Leider ist es praktisch nicht umsetzbar, in einer Hardwarelösung
    zu unterscheiden, welcher Buffer Overflow böswillig und welcher gewollt ist. 
    Damit ist dieser Ansatz nicht praktikabel.
\end{frame}

\begin{frame}{Gegenmaßnahmen}{C Range Error Detector und Out Of Bounds Object}
    \begin{itemize}
        \item Out Of Bounds Object hilft Referenzen ungefährlich zu machen.
        \item Jede Adresse, welche nicht im gültigen Bereich liegt, verweist auf das OOBO. 
        \item Kann umgangen werden, sofern der Angreifer über Speicheradressen informiert ist. %die Adressen des Programms
    \end{itemize}
\end{frame}




\begin{frame}{Gegenmaßnahmen}{Testen}
    \begin{itemize}
        \item Erhöhen Sicherheit und Robustheit
        \item Fuzzy Tests -- gezielt zufällig
        \item Spezifische Payloads und Escape-Sequenzen
        \item Werkzeuge wie SonarLint analysieren statisch% können häufig auftretende Fehler entdeckt werden.
    \end{itemize}    
\end{frame}


\begin{frame}{Gegenmaßnahmen}{Statische Analyse}

\begin{itemize}
    \item Programm wird bereits vor der Ausführung analysiert 
    \item Kann Schwachstellen bestimmen und Lösungen vorschlagen
    \item Zeitsparend aber alleine nicht ausreichend.
    \item Abwesenheit von Schwachstellen unmöglich zu bestimmen.    
\end{itemize}

\end{frame}


\begin{frame}{Gegenmaßnahmen}{Bug-Bounty-Programme} 
    Externalisieren von Tests durch Prämien für gefundene Sicherheitslücken.

    \vspace{1em}
    Die Prämien sind davon abhängig:
    \begin{itemize}
        \item in welcher Version der Software
        \item unter welchen Voraussetzungen
    \end{itemize}

    eine Schwachstelle vorliegt und welche Art des Zugriffes der Angriff ermöglicht.

    \vspace{1em}
    Buffer-Overflow-Sicherheitslücken haben eine implizit hohe Zugriffsstufe \\
    und gehören dadurch zu den lukrativsten.
    %Wurde ein Buffer Overflow entdeckt, so wird meist die höchste Prämienstufe ausgezahlt, da bei einem
    %Buffer Overflow mit der passenden Payload ein Shellcode ausgeführt werden kann, welcher dann das
    %höchste Ziel bei einem Angriff - eine Root Shell - erreicht.
    %Diese hohe Zugriffsstufe macht einen Buffer-Overflow-Angriff immer zu einem der attraktivsten Vektoren. 
    %Sind mehrere Schwachstellen für ein Programm bekannt, ist dessen Version meist veraltet. 
    
    
\end{frame}


\begin{frame}{Gegenmaßnahmen}{Stack-Schutz mit Canaries} 
    (engl. Kanarienvogel) -- Kanarienvögel als
    Indikator für Gas in Minen. 
    
    %TODO: Inhalt
    
\end{frame}


\begin{frame}{Gegenmaßnahmen}{Manuelles Buffer-Overflow-Blocken}    
    Input Sanitization $\rightarrow$ infizierte Eingaben bereinigen\\ 
    Im schlimmsten Fall wird eine semantisch inkorrekte Eingabe weiterverarbeitet,\\
    nicht aber eine böswillige, durch welche Datenlecks oder ungewollte Aufrufe entstehen würden.
    \vspace{1em}
    \begin{itemize}
        \item Eingaben zuerst filtern und dann validieren.
        \item Nicht immer möglich
        \item Zwischen normaler und böswilliger Anfrage unterscheiden
        \item Nicht immer zuverlässig 
    \end{itemize} 
    \vspace{1em}
    Dieser Ansatz ist sehr effektiv, aber dafür auch sehr aufwändig.    
\end{frame}

