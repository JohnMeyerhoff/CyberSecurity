\begin{frame}{Fazit}    
    Unsere Ergebnisse zeigen, dass:
    \vspace{1em}
    \begin{itemize}
        \item es schwierig ist, sämtliche Vektoren zu überblicken. 
        \item Verfahren wie ASLR Angriffsflächen verkleinern können.
        \item absoluter Schutz bei Ansprüchen an Funktionalität, nicht möglich ist.
        \item Canaries effektiv sind.
        \item erfolgreiche Angriffe potentiell fatale (auch langfristige) folgen haben. 
    \end{itemize}   
    
  %  Eine der wenigen effektiven Maßnahmen gegen Buffer Overflows ist die Verwendung von Canaries 
  %  Diese schützen den Stack und damit die Integrität des Programms und werden verwendet, um Eingriffe zu bemerken und Schäden zu verhindern - meist durch Neustart
  %  des Prozesses bzw. Programms.

\end{frame}

\begin{frame}{Ausblick}
    Für Entwickler sind saubere und sichere Programmierfertigkeiten unabdinglich,
    um eine resistente Grundlage für robuste Software zu schaffen. 
    Vorerst bleiben Buffer Overflows eine allgegenwärtige Sicherheitslücke in vielen Programmen und Systemen.    
\end{frame}
